\documentclass[11pt,letterpaper]{article}

% ============================================
% MUHLE LAB CMS GUIDE
% A comprehensive guide for managing the lab website
% ============================================

% Packages
\usepackage[utf8]{inputenc}
\usepackage[T1]{fontenc}
\usepackage{lmodern}
\usepackage[margin=1in]{geometry}
\usepackage{graphicx}
\usepackage{xcolor}
\usepackage{hyperref}
\usepackage{fancyhdr}
\usepackage{enumitem}
\usepackage{tcolorbox}
\usepackage{booktabs}
\usepackage{longtable}
\usepackage{titlesec}
\usepackage{multicol}
\usepackage{framed}
\usepackage{tabularx}
\usepackage{float}

% ============================================
% COLOR SCHEME (Lab Branding)
% ============================================
\definecolor{labdark}{HTML}{1E1926}
\definecolor{labpurple}{HTML}{9A89B4}
\definecolor{labdate}{HTML}{389CB7}
\definecolor{linkblue}{HTML}{0066CC}
\definecolor{successgreen}{HTML}{28A745}
\definecolor{warningred}{HTML}{DC3545}

% ============================================
% HYPERREF SETTINGS
% ============================================
\hypersetup{
    colorlinks=true,
    linkcolor=labdark,
    urlcolor=linkblue,
    citecolor=labpurple,
    pdftitle={Muhle Lab CMS Guide},
    pdfauthor={Muhle Lab},
    pdfsubject={Website Content Management},
    pdfkeywords={TinaCMS, Next.js, Lab Website}
}

% ============================================
% HEADER AND FOOTER
% ============================================
\pagestyle{fancy}
\fancyhf{}
\fancyhead[L]{\textcolor{labpurple}{\small Muhle Lab CMS Guide}}
\fancyhead[R]{\textcolor{labpurple}{\small\nouppercase{\leftmark}}}
\fancyfoot[C]{\textcolor{gray}{\thepage}}
\fancyfoot[R]{\textcolor{gray}{\small muhlelab.org}}
\renewcommand{\headrulewidth}{0.5pt}
\renewcommand{\footrulewidth}{0.5pt}
\setlength{\headheight}{14pt}

% ============================================
% CUSTOM BOXES
% ============================================
\tcbuselibrary{skins,breakable}

\newtcolorbox{tipbox}{
    enhanced,
    breakable,
    colback=labpurple!10,
    colframe=labpurple,
    title={\textbf{Tip}},
    fonttitle=\bfseries\color{white},
    coltitle=white,
    attach boxed title to top left={yshift=-2mm, xshift=5mm},
    boxed title style={colback=labpurple}
}

\newtcolorbox{warningbox}{
    enhanced,
    breakable,
    colback=warningred!10,
    colframe=warningred,
    title={\textbf{Important}},
    fonttitle=\bfseries\color{white},
    coltitle=white,
    attach boxed title to top left={yshift=-2mm, xshift=5mm},
    boxed title style={colback=warningred}
}

\newtcolorbox{stepbox}[1][]{
    enhanced,
    breakable,
    colback=labdate!10,
    colframe=labdate,
    left=8pt,
    right=8pt,
    top=6pt,
    bottom=6pt,
    #1
}

\newtcolorbox{quickref}{
    enhanced,
    colback=labdark!5,
    colframe=labdark,
    title={\textbf{Quick Reference}},
    fonttitle=\bfseries\color{white},
    coltitle=white,
    attach boxed title to top center={yshift=-2mm},
    boxed title style={colback=labdark}
}

% ============================================
% TITLE FORMATTING
% ============================================
\titleformat{\section}
{\Large\bfseries\color{labdark}}{\thesection}{1em}{}[\titlerule]
\titleformat{\subsection}
{\large\bfseries\color{labpurple}}{\thesubsection}{1em}{}
\titleformat{\subsubsection}
{\normalsize\bfseries\color{labdate}}{\thesubsubsection}{1em}{}

\titlespacing*{\section}{0pt}{20pt}{10pt}
\titlespacing*{\subsection}{0pt}{15pt}{8pt}
\titlespacing*{\subsubsection}{0pt}{10pt}{5pt}

% ============================================
% DOCUMENT
% ============================================
\begin{document}

% ============================================
% TITLE PAGE
% ============================================
\begin{titlepage}
\centering
\vspace*{2cm}

{\Huge\bfseries\textcolor{labdark}{Muhle Lab}\par}
\vspace{0.5cm}
{\LARGE\textcolor{labpurple}{Website Content Management Guide}\par}

\vspace{1.5cm}

\begin{tcolorbox}[
    enhanced,
    colback=labpurple!10,
    colframe=labpurple,
    width=0.8\textwidth,
    arc=5mm,
    boxrule=1pt
]
\centering
\large A comprehensive guide for editing content on the Muhle Lab website using TinaCMS --- designed for researchers with no coding experience.
\end{tcolorbox}

\vspace{2cm}

\begin{tabular}{ll}
\textbf{Live Website:} & \url{https://muhlelab.org} \\[5pt]
\textbf{Admin Panel:} & \url{https://muhlelab.org/admin} \\[5pt]
\textbf{Repository:} & \url{https://github.com/davidfrivas/muhle-lab-web} \\
\end{tabular}

\vspace{2cm}

{\large New York State Psychiatric Institute\par}
{\large Columbia University Department of Psychiatry\par}

\vfill

\begin{tabular}{c}
\hline
\\
{\small Version 2.0 --- \today} \\
\\
\hline
\end{tabular}

\end{titlepage}

% ============================================
% TABLE OF CONTENTS
% ============================================
\tableofcontents
\newpage

% ============================================
% QUICK REFERENCE CARD (Page 2)
% ============================================
\section*{Quick Reference Card}
\addcontentsline{toc}{section}{Quick Reference Card}

\begin{multicols}{2}

\subsection*{Key URLs}
\begin{tabular}{@{}ll@{}}
\textbf{Live Site:} & muhlelab.org \\
\textbf{Admin:} & muhlelab.org/admin \\
\textbf{GitHub:} & github.com/davidfrivas/muhle-lab-web \\
\end{tabular}

\vspace{10pt}

\subsection*{Common Tasks}
\begin{tabular}{@{}lp{5cm}@{}}
\textbf{Task} & \textbf{Location} \\
\midrule
Add team member & Admin $\rightarrow$ Team Members \\
Post news & Admin $\rightarrow$ News Posts \\
Update research & Admin $\rightarrow$ Research Projects \\
Edit lab info & Admin $\rightarrow$ Site Settings \\
Add alumni & Admin $\rightarrow$ Alumni \\
Update funding & Admin $\rightarrow$ Funding Sources \\
\end{tabular}

\columnbreak

\subsection*{Keyboard Shortcuts}
\begin{tabular}{@{}ll@{}}
\textbf{Ctrl/Cmd + S} & Save \\
\textbf{Ctrl/Cmd + B} & Bold \\
\textbf{Ctrl/Cmd + I} & Italic \\
\textbf{Ctrl/Cmd + K} & Insert Link \\
\textbf{Ctrl/Cmd + Z} & Undo \\
\textbf{Ctrl/Cmd + Shift + Z} & Redo \\
\end{tabular}

\vspace{10pt}

\subsection*{Image Requirements}
\begin{tabular}{@{}ll@{}}
\textbf{Profile photos:} & 400$\times$400px (square) \\
\textbf{News featured:} & 1200$\times$700px (landscape) \\
\textbf{Banners:} & 2000$\times$800px minimum \\
\textbf{Format:} & JPEG or PNG \\
\textbf{Max size:} & 5MB \\
\end{tabular}

\end{multicols}

\begin{quickref}
\textbf{Display Order for Team Members:}
\begin{multicols}{3}
\begin{itemize}[leftmargin=*, noitemsep]
    \item[\textbf{1}] Principal Investigator
    \item[\textbf{2--5}] Postdocs / Senior Staff
    \item[\textbf{6--10}] Research Assistants
    \item[\textbf{11--20}] Graduate Students
    \item[\textbf{21+}] Undergraduates
\end{itemize}
\end{multicols}
\end{quickref}

\newpage

% ============================================
% CHAPTER 1: INTRODUCTION
% ============================================
\section{Introduction}

Welcome to the Muhle Lab website content management system. This guide will walk you through how to add, edit, and manage content on the lab website without any coding knowledge.

\subsection{What is TinaCMS?}

TinaCMS is a visual content management system that allows you to edit website content directly through your web browser. Think of it as a word processor for your website.

\begin{tipbox}
\textbf{No coding required!} If you can use Microsoft Word or Google Docs, you can use TinaCMS.
\end{tipbox}

\textbf{Key Benefits:}
\begin{itemize}
    \item \textbf{Visual Editing} --- See your changes as you make them
    \item \textbf{Automatic Saving} --- Changes are saved to GitHub automatically
    \item \textbf{Fast Updates} --- Website updates within 2--3 minutes of saving
    \item \textbf{Version History} --- All changes are tracked and can be reverted
    \item \textbf{Secure} --- Only authorized collaborators can make changes
\end{itemize}

\subsection{Getting Access}

To edit the website, you need:
\begin{enumerate}
    \item A GitHub account (free at \url{https://github.com})
    \item To be added as a collaborator on the repository
\end{enumerate}

\begin{warningbox}
Contact the lab webmaster to be added as a collaborator if you don't already have access.
\end{warningbox}

\subsection{Accessing the Editor}

\begin{stepbox}
\textbf{Step 1:} Open your web browser (Chrome, Firefox, Safari, or Edge)
\end{stepbox}

\begin{stepbox}
\textbf{Step 2:} Go to \textbf{\url{https://muhlelab.org/admin}}
\end{stepbox}

\begin{stepbox}
\textbf{Step 3:} Click \textbf{``Log in with GitHub''}
\end{stepbox}

\begin{stepbox}
\textbf{Step 4:} If prompted, authorize TinaCMS to access your GitHub account
\end{stepbox}

\begin{stepbox}
\textbf{Step 5:} You will be redirected to the visual editor dashboard
\end{stepbox}

\newpage

% ============================================
% CHAPTER 2: NAVIGATING THE EDITOR
% ============================================
\section{Navigating the Editor}

\subsection{The Main Dashboard}

When you first log in, you'll see the main dashboard with a sidebar listing all content types:

\begin{center}
\begin{tabular}{lp{8cm}}
\toprule
\textbf{Content Type} & \textbf{Description} \\
\midrule
Team Members & Current lab members with bios and photos \\
Alumni & Former lab members \\
News Posts & Lab news, announcements, and events \\
Research Projects & Research aims and ongoing projects \\
Funding Sources & Grants and funding information \\
Site Settings & Global website settings (logo, contact info) \\
\bottomrule
\end{tabular}
\end{center}

\subsection{Content List View}

Click on any content type to see a list of existing items. From here you can:
\begin{itemize}
    \item \textbf{Click on an item} to edit it
    \item \textbf{Click ``Create New''} to add a new item
    \item \textbf{Use the search bar} to find specific items
\end{itemize}

\subsection{The Edit Form}

When editing an item, you'll see a form with various fields:

\begin{center}
\begin{tabular}{lp{9cm}}
\toprule
\textbf{Field Type} & \textbf{How to Use} \\
\midrule
Text & Type a single line of text \\
Rich Text & Formatted text with bold, italic, links, lists \\
Image & Click to upload or select an image \\
Date & Click to open a calendar picker \\
Toggle & Click to switch on/off (e.g., Published) \\
Dropdown & Click to select from options \\
Number & Enter a numeric value \\
\bottomrule
\end{tabular}
\end{center}

\subsection{Saving Changes}

\begin{warningbox}
Always click \textbf{``Save''} when you're done editing. Unsaved changes will be lost if you navigate away!
\end{warningbox}

After saving:
\begin{enumerate}
    \item Your changes are committed to GitHub automatically
    \item The website rebuilds (takes 2--3 minutes)
    \item Changes appear on the live site
\end{enumerate}

\newpage

% ============================================
% CHAPTER 3: MANAGING TEAM MEMBERS
% ============================================
\section{Managing Team Members}

\subsection{Adding a New Team Member}

\begin{stepbox}
\textbf{Step 1:} Navigate to \textbf{Team Members} in the sidebar
\end{stepbox}

\begin{stepbox}
\textbf{Step 2:} Click the \textbf{``Create New''} button
\end{stepbox}

\begin{stepbox}
\textbf{Step 3:} Fill in the required fields:
\begin{itemize}[noitemsep]
    \item \textbf{Full Name} --- e.g., ``Jane Smith''
    \item \textbf{Credentials} --- e.g., ``Ph.D.'' or leave blank
    \item \textbf{Role} --- Select from the dropdown menu
    \item \textbf{Profile Photo} --- Upload a square image
    \item \textbf{Display Order} --- Number for position on page (lower = first)
\end{itemize}
\end{stepbox}

\begin{stepbox}
\textbf{Step 4:} Write the biography using the rich text editor
\end{stepbox}

\begin{stepbox}
\textbf{Step 5:} (Optional) Add social media links (email, Twitter, LinkedIn, etc.)
\end{stepbox}

\begin{stepbox}
\textbf{Step 6:} Click \textbf{``Save''}
\end{stepbox}

\subsection{Profile Photo Requirements}

\begin{center}
\begin{tabular}{ll}
\toprule
\textbf{Requirement} & \textbf{Value} \\
\midrule
Format & JPEG or PNG \\
Minimum Size & 400 $\times$ 400 pixels \\
Aspect Ratio & Square (1:1) \\
Maximum File Size & 5MB \\
\bottomrule
\end{tabular}
\end{center}

\begin{tipbox}
For best results, crop photos to a square before uploading. Photos should be professional headshots with good lighting.
\end{tipbox}

\subsection{Writing Biographies}

\textbf{Best Practices:}
\begin{itemize}
    \item Keep biographies to 1--2 paragraphs
    \item Include educational background
    \item Describe current research focus
    \item Mention any personal interests (optional)
    \item Use italics for gene names (e.g., \emph{CHD8}, \emph{Chd8})
\end{itemize}

\begin{tipbox}
\textbf{Formatting gene names:} Highlight the gene name text and click the \emph{I} (italic) button, or press \textbf{Ctrl/Cmd + I}.
\end{tipbox}

\subsection{Display Order}

The display order determines where members appear on the Team page:

\begin{center}
\begin{tabular}{cl}
\toprule
\textbf{Order} & \textbf{Typical Use} \\
\midrule
1 & Principal Investigator \\
2--5 & Senior staff and Postdocs \\
6--10 & Research Assistants \\
11--20 & Graduate Students \\
21+ & Undergraduate Researchers \\
\bottomrule
\end{tabular}
\end{center}

\subsection{Moving Members to Alumni}

When a team member leaves the lab:

\begin{enumerate}
    \item Open their Team Members entry and note their information
    \item Delete their Team Members entry
    \item Go to \textbf{Alumni} and click \textbf{``Create New''}
    \item Fill in their information, including:
    \begin{itemize}
        \item Years active in the lab
        \item Their current position (if known)
        \item Whether to feature them prominently
    \end{itemize}
    \item Click \textbf{``Save''}
\end{enumerate}

\newpage

% ============================================
% CHAPTER 4: PUBLISHING NEWS
% ============================================
\section{Publishing News}

\subsection{When to Post}

Good topics for lab news include:
\begin{itemize}
    \item Conference presentations and posters
    \item Awards and honors
    \item New grants and funding
    \item New team members joining
    \item Graduations and celebrations
    \item Paper publications and preprints
    \item Research milestones
    \item Lab events and outings
\end{itemize}

\subsection{Creating a News Post}

\begin{stepbox}
\textbf{Step 1:} Navigate to \textbf{News Posts} in the sidebar
\end{stepbox}

\begin{stepbox}
\textbf{Step 2:} Click \textbf{``Create New''}
\end{stepbox}

\begin{stepbox}
\textbf{Step 3:} Fill in the basic information:
\begin{itemize}[noitemsep]
    \item \textbf{Title} --- Clear, descriptive (e.g., ``INSAR 2024 Conference'')
    \item \textbf{Publication Date} --- When to show the post
    \item \textbf{Featured Image} --- Main image for the post
    \item \textbf{Image Description} --- Alt text for accessibility
\end{itemize}
\end{stepbox}

\begin{stepbox}
\textbf{Step 4:} Write the post content using the rich text editor
\end{stepbox}

\begin{stepbox}
\textbf{Step 5:} (Optional) Add carousel images for events with multiple photos
\end{stepbox}

\begin{stepbox}
\textbf{Step 6:} Set publishing options:
\begin{itemize}[noitemsep]
    \item Check \textbf{``Published''} to make visible on the site
    \item Check \textbf{``Featured on Homepage''} to show in the latest news section
\end{itemize}
\end{stepbox}

\begin{stepbox}
\textbf{Step 7:} Click \textbf{``Save''}
\end{stepbox}

\subsection{Image Guidelines}

\begin{center}
\begin{tabular}{lll}
\toprule
\textbf{Image Type} & \textbf{Recommended Size} & \textbf{Orientation} \\
\midrule
Featured Image & 1200 $\times$ 700 pixels & Landscape \\
Carousel Images & 800 $\times$ 800 pixels & Square \\
\bottomrule
\end{tabular}
\end{center}

\begin{tipbox}
Include people in your images when possible --- they're more engaging than text or logos alone!
\end{tipbox}

\subsection{Image Carousels}

For events with multiple photos (conferences, parties, etc.):

\begin{enumerate}
    \item Scroll to the \textbf{``Image Carousel''} section
    \item Click \textbf{``Add Image''} for each photo
    \item Upload the image
    \item Add a description for each image
    \item Repeat for all photos (5--10 maximum recommended)
\end{enumerate}

\newpage

% ============================================
% CHAPTER 5: RESEARCH & FUNDING
% ============================================
\section{Research \& Funding}

\subsection{Research Projects}

Each research project on the Research page includes:

\begin{center}
\begin{tabular}{lp{8cm}}
\toprule
\textbf{Field} & \textbf{Description} \\
\midrule
Heading & The research question or aim title \\
Description & Rich text explaining the research \\
Figure & Supporting image with caption \\
Layout & Image position (left or right of text) \\
Order & Position on the page \\
\bottomrule
\end{tabular}
\end{center}

\begin{tipbox}
Alternate the layout (image left, then image right) for a more visually interesting page.
\end{tipbox}

\subsection{Funding Sources}

Each funding entry includes:

\begin{center}
\begin{tabular}{lp{8cm}}
\toprule
\textbf{Field} & \textbf{Description} \\
\midrule
Project Title & Official grant title \\
Program Title & Funding program name \\
Funding Source & Organization name (NIH, NIMH, etc.) \\
Logo & Organization logo image \\
Principal Investigator & PI name \\
Description & Grant description and goals \\
Status & Active or Past \\
\bottomrule
\end{tabular}
\end{center}

\begin{tipbox}
Past funding automatically appears in a separate ``Past Funding'' section at the bottom of the Funding page.
\end{tipbox}

\newpage

% ============================================
% CHAPTER 6: SITE SETTINGS
% ============================================
\section{Site Settings}

Access global settings via \textbf{Admin $\rightarrow$ Site Settings}.

\subsection{Lab Information}

\begin{itemize}
    \item \textbf{Lab Name} --- Displayed in header and footer
    \item \textbf{Tagline} --- Appears on homepage banner
    \item \textbf{Mission Statement} --- Homepage section content
\end{itemize}

\subsection{Contact Information}

\begin{itemize}
    \item \textbf{Email} --- Contact email for the lab
    \item \textbf{Address} --- Physical location
    \item \textbf{Form Endpoint} --- FormSubmit.co endpoint URL
\end{itemize}

\subsection{Banner Images}

Each page can have its own banner image:

\begin{itemize}
    \item Home page banner
    \item Team page banner
    \item Research page banner
    \item News page banner
    \item Contact page banner
\end{itemize}

\textbf{Recommended banner size:} 2000 $\times$ 800 pixels minimum

\newpage

% ============================================
% CHAPTER 7: FORMATTING TEXT
% ============================================
\section{Formatting Text}

\subsection{Basic Formatting}

The rich text editor supports:

\begin{center}
\begin{tabular}{llc}
\toprule
\textbf{Format} & \textbf{How to Apply} & \textbf{Shortcut} \\
\midrule
Bold & Click \textbf{B} button & Ctrl/Cmd + B \\
Italic & Click \emph{I} button & Ctrl/Cmd + I \\
Heading & Select from dropdown & --- \\
Bullet List & Click bullet icon & --- \\
Numbered List & Click number icon & --- \\
Link & Click link icon & Ctrl/Cmd + K \\
\bottomrule
\end{tabular}
\end{center}

\subsection{Creating Links}

\begin{enumerate}
    \item Highlight the text you want to link
    \item Click the link icon in the toolbar (or press Ctrl/Cmd + K)
    \item Enter the full URL (including \texttt{https://})
    \item Click \textbf{``Add''} or press Enter
\end{enumerate}

\subsection{Internal Links}

For links to other pages on the Muhle Lab website, use relative paths:

\begin{center}
\begin{tabular}{ll}
\texttt{/contact} & Links to the Contact page \\
\texttt{/news} & Links to the News page \\
\texttt{/team} & Links to the Team page \\
\texttt{/research} & Links to the Research page \\
\end{tabular}
\end{center}

\subsection{Gene Names}

Scientific convention requires gene names to be italicized:

\begin{itemize}
    \item Human genes: \emph{CHD8}, \emph{ARID1B}
    \item Mouse genes: \emph{Chd8}, \emph{Arid1b}
\end{itemize}

To italicize: highlight the gene name and press \textbf{Ctrl/Cmd + I}.

\newpage

% ============================================
% CHAPTER 8: TROUBLESHOOTING
% ============================================
\section{Troubleshooting}

\subsection{Common Issues}

\begin{center}
\begin{longtable}{p{4cm}p{8cm}}
\toprule
\textbf{Problem} & \textbf{Solution} \\
\midrule
\endhead
Can't log in & Verify you're a repository collaborator. Contact the webmaster to be added. \\
\midrule
Changes not showing & Wait 2--3 minutes for the site to rebuild. Try refreshing with Ctrl/Cmd + Shift + R. \\
\midrule
Build failed & Check GitHub Actions tab for error details. Usually caused by missing required fields. \\
\midrule
Image won't upload & Compress the image, check the format (JPEG/PNG only), and try renaming the file. \\
\midrule
Formatting looks wrong & Clear the formatting and reapply it. Avoid copying from Word. \\
\midrule
Site is down & Contact the webmaster immediately. \\
\bottomrule
\end{longtable}
\end{center}

\subsection{Checking Build Status}

To see if your changes are live:

\begin{enumerate}
    \item Go to the GitHub repository
    \item Click the \textbf{``Actions''} tab
    \item Look for the most recent workflow run
    \item Green checkmark = success (changes are live)
    \item Yellow circle = in progress (wait a moment)
    \item Red X = failed (click for details)
\end{enumerate}

\subsection{Getting Help}

\begin{itemize}
    \item \textbf{Lab Webmaster} --- For most issues and access requests
    \item \textbf{GitHub Issues} --- Open an issue on the repository for bugs
    \item \textbf{TinaCMS Docs} --- \url{https://tina.io/docs/}
    \item \textbf{Documentation} --- Check the \texttt{docs/} folder in the repository
\end{itemize}

\newpage

% ============================================
% APPENDIX A: KEYBOARD SHORTCUTS
% ============================================
\appendix
\section{Keyboard Shortcuts}

\begin{center}
\begin{tabular}{ll}
\toprule
\textbf{Shortcut} & \textbf{Action} \\
\midrule
Ctrl/Cmd + S & Save changes \\
Ctrl/Cmd + B & Bold text \\
Ctrl/Cmd + I & Italic text \\
Ctrl/Cmd + K & Insert link \\
Ctrl/Cmd + Z & Undo \\
Ctrl/Cmd + Shift + Z & Redo \\
Ctrl/Cmd + Shift + R & Hard refresh browser \\
\bottomrule
\end{tabular}
\end{center}

% ============================================
% APPENDIX B: RESOURCES
% ============================================
\section{Resources}

\subsection{Quick Links}

\begin{itemize}
    \item \textbf{Live Website:} \url{https://muhlelab.org}
    \item \textbf{Admin Panel:} \url{https://muhlelab.org/admin}
    \item \textbf{GitHub Repository:} \url{https://github.com/davidfrivas/muhle-lab-web}
    \item \textbf{TinaCMS Dashboard:} \url{https://app.tina.io}
\end{itemize}

\subsection{Documentation}

All documentation is available in the \texttt{docs/} folder:

\begin{itemize}
    \item \texttt{GETTING\_STARTED.md} --- First-time setup
    \item \texttt{ADDING\_TEAM\_MEMBERS.md} --- Team member guide
    \item \texttt{PUBLISHING\_NEWS.md} --- News posting guide
    \item \texttt{EDITING\_CONTENT.md} --- General editing tips
    \item \texttt{TROUBLESHOOTING.md} --- Common issues
    \item \texttt{CUSTOMIZATION\_GUIDE.md} --- For other labs using this template
\end{itemize}

\vfill

\begin{center}
\rule{0.5\textwidth}{0.5pt}

\vspace{10pt}

{\large\textbf{Muhle Lab}}

New York State Psychiatric Institute

Columbia University Department of Psychiatry

\vspace{10pt}

\textcolor{gray}{\small This guide was created for non-technical researchers to easily manage lab website content.}
\end{center}

\end{document}
